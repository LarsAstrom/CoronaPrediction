\documentclass[a4paper, 12pt]{article}
\setcounter{secnumdepth}{2}
\setcounter{tocdepth}{2}

%% Språk och font
\usepackage[swedish]{babel}
%\usepackage[utf8x]{inputenc}
\usepackage[T1]{fontenc}
\usepackage[utf8]{inputenc}
\usepackage{biblatex}
\usepackage{dirtytalk}

%% Sätter pappersstorlek och marginaler
\usepackage[a4paper,top=2.5cm,bottom=2.5cm,left=2.5cm,right=2.5cm,marginparwidth=1.75cm]{geometry}

%% Gör att figurer namnges efter kapitel
\usepackage{chngcntr}
\counterwithin{figure}{section}

%% Användbara paket
\usepackage{amsmath}
\usepackage{amssymb}
\usepackage{graphicx}
\usepackage{subcaption}
\usepackage{caption}
\usepackage{float}
\usepackage[colorinlistoftodos]{todonotes}
\usepackage[colorlinks=true, allcolors=black]{hyperref}
\usepackage{enumerate}
\usepackage{amsthm}
\usepackage{textcomp}
\usepackage{gensymb}
\usepackage{csquotes}
\let\micro\micro
\let\perthousand\perthousand
\usepackage{listings}
\lstset{
  basicstyle=\ttfamily,
  columns=fullflexible,
  frame=single,
  breaklines=true,
  postbreak=\mbox{\textcolor{red}{$\hookrightarrow$}\space},
}

%% Sats-environment
\theoremstyle{definition}
\newtheorem{exmp}{Exempel}[section]
\newtheorem*{lsn}{Lösning}
\newtheorem{sats}{Sats}[section]
\newtheorem{definition}{Definition}[section]
\newtheorem*{anm}{Anmärkning}
\newtheorem{uppg}{}[subsection]

%% Figurer
\usepackage{siunitx}
\setlength{\parindent}{0pt}
\setlength{\parskip}{10pt}
\setlength{\fboxsep}{.5\fboxsep}
\newcommand{\mrel}{\mathrel{\bigcirc}}

%% För tabeller
\usepackage{array}
\usepackage{tabularx}
\newcolumntype{L}[1]{>{\raggedright\let\newline\\\arraybackslash\hspace{0pt}}m{#1}}
\newcolumntype{C}[1]{>{\centering\let\newline\\\arraybackslash\hspace{0pt}}m{#1}}
\newcolumntype{R}[1]{>{\raggedleft\let\newline\\\arraybackslash\hspace{0pt}}m{#1}}

%% Titel och författare
\title{Instruktioner för att köra Corona virus modell}
\author{Lars Åström}
\date{April 2020} %För inget datum

%\linespread{1.25}, for increased spacing

\begin{document}
\maketitle
\section{Introduktion}
Målet med detta dokument är att förklara hur programmen körs, samt att göra så att användaren kan få en prediktionsgraf i Excel över antalet inlagda, sjuka, immuna, döda och ej ännu infekterade individer.

\section{Instruktioner för Windows}
\subsection{Få aktuella parametrar och prediktioner}
\begin{enumerate}
  \item Se till att python är installerat och att IDLE finns. Instruktioner för att ladda ner detta finns i appendix.
  \item Om ny data har kommit, lägg in denna. Detta behöver göras på två ställen, både i input.csv som ligger i mappen input\_data och under fliken ''Data'' i simulation.xlsx
  \item Öppna Python IDLE. Detta kan göras genom att klicka på Windows-key och därefter skriva IDLE.
  \item Öppna filen corona\_prediktion\_idag.py genom att klicka på Ctrl+O och därefter går till där filen finns och sedan klicka på Öppna.
  \item En ny ruta kommer att öppnas med koden i. Ändra inte i koden utan kör programmet genom att klicka på F5. Detta kommando kommer att ta några minuter att bli klart.
  \item När programmet är klart kommer det stå vad parametrarna ska vara för att optimera anpassningen.
  \item Öppna excel-filen simulation.xlsx. Gå till fliken ''Konfiguration'' och ändra parametrarna alpha, beta, gamma och L. Obs! Ändra inte K, den kommer att uppdateras av sig själv.
  \item När detta har gjorts kommer två saker att uppdateras. Först och främst ändras grafen som finns under fliken ''Konfiguration''. Dessutom kommer information om toppen, eller peaken, att uppdateras i resultatrutan under samma flik.
\end{enumerate}

\subsection{Kolla hur parametrarna förändrats över tid}
\begin{enumerate}
  \item Se till att python är installerat och att IDLE finns. Instruktioner för att ladda ner detta finns i appendix.
  \item Om ny data har kommit, lägg in denna. Detta behöver göras i input.csv.
  \item Öppna Python IDLE. Detta kan göras genom att klicka på Windows-key och därefter skriva IDLE.
  \item Öppna filen corona\_parameter\_tidsserie.py genom att klicka på Ctrl+O och därefter går till där filen finns och sedan klicka på Öppna.
  \item En ny ruta kommer att öppnas med koden i. Ändra inte i koden utan kör programmet genom att klicka på F5. Detta kommando kommer att ta 15 minuter att bli klart.
  \item Nästa gång programmet körs kommer det gå mycket snabbare. Anledningen är att resultaten sparas i log\_parameter\_tidsserie.csv. Dock, om man vill att allt ska beräknas om så behöver man tömma denna filen.
\end{enumerate}

\section{Appendix}
\subsection{Installera python och IDLE}
\begin{enumerate}
  \item Gå till hemsidan python.org.
  \item Klicka på ''Downloads'' och därefter Ladda ner.
  \item Följ stegen som krävs för att python ska bli installerat. Klicka bara på nästa och låt installationsprogrammet installera allt som är förhandsvalt.
\end{enumerate}

\end{document}

\iffalse
%% För tre figurer med samma siffra.
\begin{figure}[H]
\centering
\begin{subfigure}[b]{.3\textwidth}
\centering
\includegraphics[width=.99hsize]{AG1RatLinje_HittaM1lsn.png}
\caption{}\label{Fig: AG1RatLinje HittaM1lsn}
\end{subfigure}
\begin{subfigure}[b]{.3\textwidth}
\centering
\includegraphics[width=.9\hsize]{AG1RatLinje_HittaM2lsn.png}
\caption{}
\end{subfigure}
\begin{subfigure}[b]{.3\textwidth}
\centering
\includegraphics[width=.9\hsize]{AG1RatLinje_HittaM3lsn.png}
\caption{}\label{Fig: AG1RatLinje HittaM3lsn}
\end{subfigure}
\caption{}
\end{figure}

%% För två figurer med olika siffror.
\begin{figure}[H]
\centering
\begin{minipage}{.5\textwidth}
\centering
\includegraphics[width=.4\linewidth]{image1}
\captionof{figure}{A figure}
\label{fig:test1}
\end{minipage}
\begin{minipage}{.5\textwidth}
\centering
\includegraphics[width=.4\linewidth]{image1}
\captionof{figure}{Another figure}
\label{fig:test2}
\end{minipage}
\end{figure}

%% För en figur.
\begin{figure}[H]
\centering
\includegraphics[width=0.6\textwidth]{xxx.png}
\caption{}\label{}
\end{figure}

%% För en figur med 3*2 subfigurer
\begin{figure}[t!] % "[t!]" placement specifier just for this example
\begin{subfigure}{0.48\textwidth}
\includegraphics[width=\linewidth]{pic1.pdf}
\caption{First subfigure} \label{fig:a}
\end{subfigure}\hspace*{\fill}
\begin{subfigure}{0.48\textwidth}
\includegraphics[width=\linewidth]{pic2.pdf}
\caption{Second subfigure} \label{fig:b}
\end{subfigure}

\medskip
\begin{subfigure}{0.48\textwidth}
\includegraphics[width=\linewidth]{pic3.pdf}
\caption{Third subfigure} \label{fig:c}
\end{subfigure}\hspace*{\fill}
\begin{subfigure}{0.48\textwidth}
\includegraphics[width=\linewidth]{pic4.pdf}
\caption{Fourth subfigure} \label{fig:d}
\end{subfigure}

\medskip
\begin{subfigure}{0.48\textwidth}
\includegraphics[width=\linewidth]{pic5.pdf}
\caption{Fifth subfigure} \label{fig:e}
\end{subfigure}\hspace*{\fill}
\begin{subfigure}{0.48\textwidth}
\includegraphics[width=\linewidth]{pic6.pdf}
\caption{Sixth subfigure} \label{fig:f}
\end{subfigure}

\caption{My complicated figure} \label{fig:1}
\end{figure}

%%För att göra matriser:
\[
\begin{bmatrix}
    x_{11} & x_{12} & x_{13} & \dots  & x_{1n} \\
    x_{21} & x_{22} & x_{23} & \dots  & x_{2n} \\
    \vdots & \vdots & \vdots & \ddots & \vdots \\
    x_{d1} & x_{d2} & x_{d3} & \dots  & x_{dn}
\end{bmatrix}
\]

%Tabell
\begin{center}
        \begin{tabular}{c|c|c|c}
            Data set & Base 1 & Base 2 & Base 3 \\ \hline
            Training data & 821 & 860 & 945 \\ \hline
            Test data & 795 & 649 & 697 \\ \hline
        \end{tabular}
    \end{center}
\fi
